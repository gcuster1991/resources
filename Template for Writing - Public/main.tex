%%%%%%%%%%%%%%%%%%%%%%%%%%%%%%%%%%%%%%%%%%%%%%%%%%%%%%%%%%%%%%%%
%% Template created by Jessi Rick (jrick@uwyo.edu), Fall 2018 %%
%%%%%%%%%%%%%%%%%%%%%%%%%%%%%%%%%%%%%%%%%%%%%%%%%%%%%%%%%%%%%%%%

%%%%%% Preamble of the document begins here %%%%%%
\documentclass[letterpaper,12pt]{article}

% load the packages that we use for building the document
\usepackage[english]{babel}
\usepackage[utf8x]{inputenc}
\usepackage{amsmath}
\usepackage{graphicx}
\usepackage{subcaption} % allows two captions for two-part figures
\usepackage[margin=1in]{geometry} %% sets 1" margins
%\renewcommand{\baselinestretch}{1.5} %% this sets to 1.5 spacing
%\usepackage{setspace} %% these two set double spacing
%\doublespacing %% goes with previous
\usepackage{booktabs} % for making tables look nice
\usepackage{latex-rmd} % a custom package for making R Markdown files look nice

\usepackage{natbib}
\bibpunct{(}{)}{;}{a}{}{,}  % citation format command for natbib

% for fancy headers
\usepackage{fancyhdr}
\pagestyle{fancy}
\usepackage[font={small,sf},format=plain,labelfont={bf,up}]{caption}
\fancyhf{}
\fancyhead[l,lo]{Name \textit{Short Title}}
\fancyhead[r,ro]{\thepage}

% write the title and author here
% You must supply your own values
\title{Title of this Work}
\author{Author Name(s)}
\date{Version: \today}

%%%%%%%%%%%%%%%%%%%%%%%%%%%%%%%%%%%%%%%%%%%%%%%
%%%%% Preamble to the document ends here %%%%%%
%%%%%%%%%%%%%%%%%%%%%%%%%%%%%%%%%%%%%%%%%%%%%%%

\begin{document}
\maketitle

% next we include the different sections. 
% Each section is included in the document
\begin{abstract}
    This is a generic writing template created to make it easier to just get started writing. You'll want to modify this to whatever format makes sense for the document that you're writing, but the goal is for it to be someplace to start.
\end{abstract}
\section*{Introduction}

This is the introduction text. And an example of how you might cite \cite{Patterson2012}. You could also just add a citation at the end of the sentence \citep{Patterson2012} or refer to a specific paper (e.g. \citealt{Patterson2012}). All of these citations are hyperlinked, so you can click on them to go to the references section.
\section*{Methods}
\subsection*{The first type of method}
This is some text explaining things about this first type of method.

\subsection*{The second subheader}

This is some more explanation of the methods.
\section*{Results}

This is the results text.
\section*{Discussion}

This is the discussion text.

% This part is for including references and citations
% you can change the bibliography style to whatever you'd like
\bibliographystyle{sysbio} % change to the citation format you want
\bibliography{bibliography} % change to the name of your .bib file

% individual sections for tables and figures if you don't want them embedded in-text
\clearpage
\section*{Tables}

% if you aren't sure how to include a table, https://www.tablesgenerator.com/ is awesome for just pasting data and generating a table
% there also are some good tips at https://texblog.org/2017/02/06/proper-tables-with-latex/

\clearpage
\section*{Figures}

%% FIGURE %%
% \begin{figure}[tb]
% \includegraphics[width=\textwidth]
% {figures/figure.png} \label{fig:label}
% \caption{Caption: this is an example of a single figure.}
% \end{figure}
%%%%

%% SIDE-BY-SIDE FIGURE %%
%\begin{figure}[ht]
%\centering
%\label{fig:maps}
%\begin{subfigure}{.5\textwidth}
%  \centering
%  \includegraphics[width=\linewidth]{figures/partA.jpeg}
%  \caption{}
%  \label{fig:partA}
%\end{subfigure}%
%\begin{subfigure}{.5\textwidth}
%  \centering
%  \includegraphics[width=\linewidth]{figures/partB.jpeg}
%  \caption{}
%  \label{fig:partB}
%\end{subfigure}
%\caption{Overall caption for the two figures.}
%\label{fig:sidebyside}
%\end{figure}

\clearpage

%% add a supplement if desired, e.g. to include supplemental figures/text, or to include code from analyses
\clearpage
\section*{Supplement}

% The document must end with this code
\end{document}
